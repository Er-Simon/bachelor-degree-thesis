% !TeX root = ../relazione.tex

\chapter{L’importanza di SQLite nell’informatica forense}


\section{Il motore di database più utilizzato al mondo}
SQLite è presente su un grandissimo numero di dispositivi e i suoi utilizzi possono variare dal conservare le informazioni delle applicazioni di messaggistica istantanea, al conservare i formati dei file utilizzati dalle applicazioni come ad esempio Photoshop Lightroom.

È supportato dalla maggior parte dei dispositivi presenti sul mercato e da quasi tutti i sistemi operativi, anche da quelli integrati. Il numero di smartphone che ne fa un uso attivo ammonta a \textbf{circa 4 miliardi}\footnote{Fonte: SQLite.org; \url{https://www.sqlite.org/mostdeployed.html}} e si stima che ci siano più di un bilione di database SQLite in uso.

La maggior parte delle applicazioni presenti su i nostri smartphone usano database SQLite, per citarne alcune delle più famose: WhatsApp Messenger, Dropbox, Firefox, Safari e Google Chrome.
 	

\section{Quanto è importante recuperare i dati?}
I browser precedentemente nominati utilizzano database SQLite della versione 3 per gestire i dati dell’utente come la cronologia, i cookie e i file scaricati, mentre l’applicazione di messaggistica istantanea WhatsApp lo utilizza per salvare le informazioni su i media scambiati, il registro delle chiamate, i contatti frequenti, i messaggi scambiati e molte altre informazioni, come le informazioni sul dispositivo utilizzato.
Inoltre, anche molte delle applicazioni preinstallate usate dagli smartphone lo usano, come ad esempio l’applicazione per la gestione dei messaggi.

\medskip

I dati detenuti nei database possono in molti casi essere di gran valore durante un’indagine forense.
Se ad esempio un sospettato di un’indagine dovesse \textbf{eliminare} la lista della chiamate, la cronologia di una chat o della navigazione internet, essendo questi dati memorizzati quasi sempre in database di tipo SQLite, potrebbero essere recuperati manualmente o con l’ausilio di software in grado di effettuarne il recupero autonomamente.
I dati acquisiti potrebbero fornire prove schiaccianti nei confronti del sospettato. Infatti, non è raro sentire notizie di persone che tramite chat compromettenti, scambi di contenuti multimediali o ricerche su internet sono stati incriminati.
