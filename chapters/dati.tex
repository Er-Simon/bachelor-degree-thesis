
\chapter{Contrastare il recupero dei dati}
SQLite per evitare che i dati rimossi possano essere recuperati, mette a disposizione una serie di direttive e di comandi che permettono in diverse modalità la sovrascrittura dei dati rimossi. Queste direttive di default sono disattivate. 
 	
\section{Le direttive}
\paragraph{auto vacuum}
Quando non è attiva, all'eliminazione dei dati dal database, la grandezza del file principale rimane invariata, le pagine inattive verranno aggiunte alla freelist e riutilizzate all'occorrenza.
Questa direttiva può operare in 3 modalità;

\begin{itemize}
\item 0 è lo stato di default, ovvero la direttiva risulta essere disattivata
\item 1 le pagine inattive verranno spostate alla fine del file e la dimensione verrà troncata ad ogni commit di una transazione
\item 2 a differenza dello stato precedente, l'operazione descritta avverrà ad ogni invocazione della direttiva incremental vacuum
\end{itemize}


Bisogna notare che questa direttiva non evita la frammentazione dell'area contenente le celle nelle pagine (i freeblock).

\paragraph{secure delete}
Quando abilitata, questa direttiva, all'occorrenza dell'eliminazione dei dati provvederà a sovrascrivere il contenuto con degli 0. Questa operazione ha un forte impatto sulle performance, ma consente di non lasciare tracce quando un contenuto viene rimosso o modificato.

Oltre ad essere abilitata supporta un ulteriore modalità chiamata fast, la quale consiste in una via di mezzo tra essere abilitata e disabilitata. Più nello specifico fa sì che venga sovrascritto il contenuto se e solo se non aumenta la quantità di operazioni di input/output, a discapitato di un maggior numero di cicli della CPU. In questa modalità verranno lasciate delle tracce nella pagine presenti nella freelist.

\section{Il comando vacuum}
Questo comando permette la ricostruzione del database nello spazio minimo richiesto. All'esecuzione verranno rimosse le pagine inattive presenti nella freelist, verranno rimossi i freeblock dall'interno dell'area contenente le celle delle pagine e verranno sovrascritte tutte quelle aree marcate come eliminate disponibili per il riuso.

