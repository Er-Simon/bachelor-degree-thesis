% !TeX root = ../relazione.tex

\chapter{Introduzione a SQLite}


\section{Cos’è SQLite?}
SQLite è una libreria software scritta nel linguaggio di programmazione C \cite{clanguage}, che implementa un \textbf{motore di database SQL}, ideata da D. Richard Hipp, ma attualmente gestita da un team di sviluppatori.

Il progetto SQLite è iniziato il 29/05/2000 e il team ha come obiettivo di supportarne lo sviluppo fino al 2050\footnote{Dati provenienti dal sito di SQLite: \url{https://www.sqlite.org/}}.
La documentazione e il codice di SQLite sono di dominio pubblico, il codice di SQLite non è Open-Contribution. 

\begin{figure}[ht]
	\centering
	\caption{Logo di SQLite}
	\label{fig:sqlitelogo}
	\includegraphics[scale=0.5]{assets/logo_sqlite}
\end{figure}

\section{Caratteristiche principali}
L’implementazione del motore del database è;
\begin{itemize}
	\item \textbf{Self-contained}: ha bisogno di pochissime dipendenze per funzionare, infatti, non richiede l’utilizzo di librerie esterne del linguaggio C, ma si limita a utilizzare le routines della libreria standard \cite{cstandardlibrary}
	\item \textbf{Serverless}: il processo che vuole accedere al database scrive e legge direttamente dal file del database sul disco, a differenza di molti altri motori di database SQL, nei quali c’è la necessità di utilizzare un processo ad-hoc
	\item \textbf{Zero-configuration}: SQLite non richiede di essere installato prima dell’utilizzo
	\item \textbf{Transactional}: le transazioni soddisfano le proprietà ACID (Atomicità, Consistenza, Isolamento e Durabilità) \cite{acid}
\end{itemize}
Nonostante il nome possa ingannare SQLite implementa tutte le funzioni di SQL, il “Lite” presente nel nome si riferisce alla ridotta grandezza del file.

SQLite viene testato molto attentamente prima di ogni rilascio ed è molto affidabile, in quanto è utilizzato da miliardi di dispositivi di tutti i tipi senza problemi da ormai quasi due decenni. 

Prima di ogni nuovo rilascio, SQLite viene testato tramite una suite di test automatizzata che esegue milioni e milioni di casi di test che coinvolgono centinaia di milioni di istruzioni SQL, al fine di provare tutti i casi di fallimento.
SQLite ha un’ alta tolleranza ai fallimenti di allocazione della memoria e agli errori di I/O del disco. Le transazioni sono ACID, infatti, anche se interrotte da crash di sistema o interruzioni di corrente, il database rimane in uno stato consistente.
Nonostante tutti i test, anche SQLite presenta dei bug, ma a differenza di altre tecnologie simili il team di sviluppo si impegna a fornire elenchi di bug noti e un changelog sulle modifiche del codice aggiornato al minuto.

\section{Confronto con i motori SQL dotati di server}
Si potrebbe pensare di poter paragonare SQLite ai DBMS \cite{dbms} dotati di server, ma sarebbe sbagliato, in quanto sono concepiti con due scopi differenti. SQLite non compete con MySQL, PostgreSQL o software simili. SQLite nasce con lo scopo di permettere a un programma di poter operare con un database senza dover essere costretto a installare un DBMS o di dover disporre di grandi capacità di memoria.

\medskip
SQLite non è adatto nelle seguenti situazioni;
\begin{itemize}

\item Se si deve permettere l'accesso alla base di dati tramite la rete internet
\item Se è necessario garantire un alto grado di concorrenza, ovvero, se dovessero essere eseguite in uno stesso istante differenti query
\item Se è richiesto un elevato utilizzo della memoria.
\end{itemize}

SQLite è particolarmente indicato nei casi in cui l'archiviazione dei dati avviene localmente nel dispositivo, la quantità dei dati non supera un terabyte di contenuto e non vengono effettuate molte operazioni nello stesso istante. 

SQLite è veloce e affidabile, non richiede configurazioni particolari o attività di manutenzione. 